\documentclass[12pt]{article}
\usepackage{pmmeta}
\pmcanonicalname{ExamplesOfNormalFormGames}
\pmcreated{2013-03-22 12:51:27}
\pmmodified{2013-03-22 12:51:27}
\pmowner{Henry}{455}
\pmmodifier{Henry}{455}
\pmtitle{examples of normal form games}
\pmrecord{9}{33192}
\pmprivacy{1}
\pmauthor{Henry}{455}
\pmtype{Example}
\pmcomment{trigger rebuild}
\pmclassification{msc}{91A10}
\pmclassification{msc}{91A06}
\pmclassification{msc}{91A05}
\pmdefines{prisoners dilemma}
\pmdefines{battle of the sexes}

% this is the default PlanetMath preamble.  as your knowledge
% of TeX increases, you will probably want to edit this, but
% it should be fine as is for beginners.

% almost certainly you want these
\usepackage{amssymb}
\usepackage{amsmath}
\usepackage{amsfonts}

% used for TeXing text within eps files
%\usepackage{psfrag}
% need this for including graphics (\includegraphics)
%\usepackage{graphicx}
% for neatly defining theorems and propositions
%\usepackage{amsthm}
% making logically defined graphics
%%%\usepackage{xypic}

% there are many more packages, add them here as you need them

% define commands here
\begin{document}
A few example normal form games:

\textbf{Prisoner's dilemma}

Probably the most famous game theory example, the prisoner's dilemma is a two player game where $S_1=S_2={C,D}$ and:
\begin{displaymath}
u_1(s_1,s_2)=\left\{
\begin{array}{ccccc}
5 &\text{ if } &s_1=C &\text{ and } &s_2=C\\
10 &\text{ if } &s_1=D &\text{ and }& s_2=C\\
-5 &\text{ if } &s_1=C &\text{ and }& s_2=D\\
0 &\text{ if } &s_1=D &\text{ and } &s_2=D
\end{array}\right.
\end{displaymath}
\begin{displaymath}
u_2(s_1,s_2)=\left\{
\begin{array}{ccccc}
5 &\text{ if } &s_1=C &\text{ and } &s_2=C\\
10 &\text{ if }& s_1=C& \text{ and }& s_2=D\\
-5 &\text{ if }& s_1=D& \text{ and }& s_2=C\\
0 &\text{ if } &s_1=D &\text{ and } &s_2=D
\end{array}\right.
\end{displaymath}

Traditionally this is interpreted as the case of two criminal partners separately being interrogated and asked to give up the other partner.  $C$ stands for cooperating (with their partners) by refusing to give up information, and $D$ stands for defecting and agreeing to testify against the partner.  The different payoffs reflect different jail sentences, ranging from nothing (+10) to a long jail sentence (-5), with amounts in between depending on the evidence against them.


The (much more convenient) normal form is:
\begin{tabular}{|c|cc|}
\hline
 &C&D\\
\hline
C&5,5&-5,10\\
D&10,-5&0,0\\
\hline
\end{tabular}

Notice that $(C,C)$ Pareto dominates $(D,D)$, however $(D,D)$ is the only Nash equilibrium.


\textbf{Battle of the Sexes}

Another traditional two player game.  The normal form is:
\begin{tabular}{|c|cc|}
\hline
 &O&F\\
\hline
O&2,1&0,0\\
F&0,0&1,2\\
\hline
\end{tabular}


\textbf{A Degenerate Example}

One more, rather pointless, example which illustrates a game where one player has no choice:

\begin{tabular}{|c|ccc|}
\hline
 &X&Y&Z\\
\hline
A&2,100&1,7&14,-5\\
\hline
\end{tabular}


\textbf{Undercut}

A game which illustrates an infinite (indeed, uncountable) strategy space.  There are two players and $S_1=S_2=\mathbb{R}^+$.

\begin{displaymath}
u_1(s_1,s_2)=\left\{
\begin{array}{ccc}
1 &\text{ if } &s_1<s_2\\
0 &\text{ if } &s_1\geq s_2
\end{array}\right.
\end{displaymath}
\begin{displaymath}
u_2(s_1,s_2)=\left\{
\begin{array}{ccc}
1 &\text{ if } & s_2<s_1\\
0 &\text{ if } & s_2\geq s_1
\end{array}\right.
\end{displaymath}
%%%%%
%%%%%
\end{document}
