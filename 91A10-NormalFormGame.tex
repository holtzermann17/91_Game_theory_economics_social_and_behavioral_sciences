\documentclass[12pt]{article}
\usepackage{pmmeta}
\pmcanonicalname{NormalFormGame}
\pmcreated{2013-03-22 12:51:24}
\pmmodified{2013-03-22 12:51:24}
\pmowner{Henry}{455}
\pmmodifier{Henry}{455}
\pmtitle{normal form game}
\pmrecord{6}{33191}
\pmprivacy{1}
\pmauthor{Henry}{455}
\pmtype{Definition}
\pmcomment{trigger rebuild}
\pmclassification{msc}{91A10}
\pmclassification{msc}{91A06}
\pmclassification{msc}{91A05}
\pmsynonym{strategic form game}{NormalFormGame}
\pmrelated{Game}
\pmdefines{normal form game}
\pmdefines{normal form}

\endmetadata

% this is the default PlanetMath preamble.  as your knowledge
% of TeX increases, you will probably want to edit this, but
% it should be fine as is for beginners.

% almost certainly you want these
\usepackage{amssymb}
\usepackage{amsmath}
\usepackage{amsfonts}

% used for TeXing text within eps files
%\usepackage{psfrag}
% need this for including graphics (\includegraphics)
%\usepackage{graphicx}
% for neatly defining theorems and propositions
%\usepackage{amsthm}
% making logically defined graphics
%%%\usepackage{xypic}

% there are many more packages, add them here as you need them

% define commands here
\begin{document}
A \emph{normal form game} is a game of complete information in which there is a list of $n$ players, numbered $1,\ldots,n$.  Each player has a strategy set, $S_i$ and a utility function $u_i:\prod_{i\leq n} S_i\rightarrow \mathbb{R}$.

In such a game each player simultaneously selects a move $s_i\in S_i$ and receives $u_i((s_1,\ldots,s_n))$.

Normal form games with two players and finite strategy sets can be represented in normal form, a matrix where the rows each stand for an element of $S_1$ and the columns for an element of $S_2$.  Each cell of the matrix contains an ordered pair which states the payoffs for each player.  That is, the cell $i,j$ contains $(u_1(s_i,s_j),u_2(s_i,s_j))$ where $s_i$ is the $i$-th element of $S_1$ and $s_j$ is the $j$-th element of $S_2$.
%%%%%
%%%%%
\end{document}
