\documentclass[12pt]{article}
\usepackage{pmmeta}
\pmcanonicalname{RubiksCube}
\pmcreated{2013-03-22 15:37:45}
\pmmodified{2013-03-22 15:37:45}
\pmowner{PrimeFan}{13766}
\pmmodifier{PrimeFan}{13766}
\pmtitle{Rubik's cube}
\pmrecord{6}{37555}
\pmprivacy{1}
\pmauthor{PrimeFan}{13766}
\pmtype{Definition}
\pmcomment{trigger rebuild}
\pmclassification{msc}{91A24}
\pmsynonym{Rubik cube}{RubiksCube}

% this is the default PlanetMath preamble.  as your knowledge
% of TeX increases, you will probably want to edit this, but
% it should be fine as is for beginners.

% almost certainly you want these
\usepackage{amssymb}
\usepackage{amsmath}
\usepackage{amsfonts}

% used for TeXing text within eps files
%\usepackage{psfrag}
% need this for including graphics (\includegraphics)
%\usepackage{graphicx}
% for neatly defining theorems and propositions
%\usepackage{amsthm}
% making logically defined graphics
%%%\usepackage{xypic}

% there are many more packages, add them here as you need them

% define commands here
\begin{document}
A mechanical puzzle in the shape of a cube, invented by the Hungarian mathematician Ernö Rubik. Each of the six faces is subdivided into nine squares, which may be painted in any of six different colors, and any of the sides may be rotated independently of the others. The central square of each face rotates with the sides, but relative to each other, the central squares are stationary.

A Rubik's cube is said to be solved when the nine squares of each face all have the same color. When any face has two or more colors on it, the cube is said to be scrambled.

Hypothesis. The maximum number of moves from a scrambled to a solved cube is 22. (A quarter rotation and a half rotation both count as one move).

It is known for a fact that for a given cube there are more than 43 quintillion possible combinations, but the maximum number of moves hypothesis remains to be proven.

Currently, the best (though not necessarily speediest) method to solve the cube is considered to be the solve-by-layers method. Instead of trying to solve one face at a time, one tries to first solve the ``top'' layer. E.g., if the solver chooses white to be the top face, and his cube has blue as the opposite color of white, the solver solves the white face so that the top three squares of the red face are all red, the top three squares of the blue face are all blue, etc. Then he solves the middle layer so that between the central squares of any two faces (except the top and bottom face) there is a piece with the appropriate matching colors. Lastly, the bottom layer is solved first by putting the corners in their place and lastly the middle pieces.

White is usually the face that the licensed manufacturer puts their logo on. The license only requires that the six colors be distinct. One popular choice of colors is white, red, green, orange, blue and yellow. In the United States, Rubik's cube is usually implemented in black plastic with colored stickers on the squares of the faces. This permits someone to solve the cube by removing the stickers and replacing them elsewhere on the cube (this is not considered a legitimate solution).

The standard Rubik's cube is sometimes referred to as 3 by 3 cube. Variations range from 2 by 2 cubes to 6 by 6 cubes.
%%%%%
%%%%%
\end{document}
