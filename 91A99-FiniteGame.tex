\documentclass[12pt]{article}
\usepackage{pmmeta}
\pmcanonicalname{FiniteGame}
\pmcreated{2013-03-22 16:35:08}
\pmmodified{2013-03-22 16:35:08}
\pmowner{PrimeFan}{13766}
\pmmodifier{PrimeFan}{13766}
\pmtitle{finite game}
\pmrecord{4}{38777}
\pmprivacy{1}
\pmauthor{PrimeFan}{13766}
\pmtype{Definition}
\pmcomment{trigger rebuild}
\pmclassification{msc}{91A99}

% this is the default PlanetMath preamble.  as your knowledge
% of TeX increases, you will probably want to edit this, but
% it should be fine as is for beginners.

% almost certainly you want these
\usepackage{amssymb}
\usepackage{amsmath}
\usepackage{amsfonts}

% used for TeXing text within eps files
%\usepackage{psfrag}
% need this for including graphics (\includegraphics)
%\usepackage{graphicx}
% for neatly defining theorems and propositions
%\usepackage{amsthm}
% making logically defined graphics
%%%\usepackage{xypic}

% there are many more packages, add them here as you need them

% define commands here

\begin{document}
A {\em finite game} is a game in which each instance of the game is designed to reach a conclusion, either as a natural consequence of normal play, or due to the rules of the game.

For example, a game of Reversi is over when all the squares of the board have been filled up, if not sooner. Most chess games end with one side trapping the opponent's king. Theoretically, certain scenarios could go on indefinitely if both sides lack the resources to trap the opponent's king; for example, if White has only a king and a knight, and Black has only the king left, Black could theoretically evade indefinitely if it weren't for the stalemate rule.
%%%%%
%%%%%
\end{document}
