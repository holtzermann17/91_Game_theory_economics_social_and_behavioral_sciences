\documentclass[12pt]{article}
\usepackage{pmmeta}
\pmcanonicalname{GameOfLife}
\pmcreated{2013-03-22 16:35:11}
\pmmodified{2013-03-22 16:35:11}
\pmowner{PrimeFan}{13766}
\pmmodifier{PrimeFan}{13766}
\pmtitle{Game of Life}
\pmrecord{5}{38778}
\pmprivacy{1}
\pmauthor{PrimeFan}{13766}
\pmtype{Definition}
\pmcomment{trigger rebuild}
\pmclassification{msc}{91A99}
\pmsynonym{Conway's Game of Life}{GameOfLife}

% this is the default PlanetMath preamble.  as your knowledge
% of TeX increases, you will probably want to edit this, but
% it should be fine as is for beginners.

% almost certainly you want these
\usepackage{amssymb}
\usepackage{amsmath}
\usepackage{amsfonts}

% used for TeXing text within eps files
%\usepackage{psfrag}
% need this for including graphics (\includegraphics)
%\usepackage{graphicx}
% for neatly defining theorems and propositions
%\usepackage{amsthm}
% making logically defined graphics
%%%\usepackage{xypic}

% there are many more packages, add them here as you need them

% define commands here

\begin{document}
The {\em Game of Life} is a cellular automaton that models a population of living organisms living on a two-dimensional plane subdivided into squares. One cell may live in each square. John Horton Conway set down the rules of the game in {\it Scientific American}:

\begin{enumerate}
\item If a cell has less than two neighbors alive in any of the eight adjacent squares (those immediately above and below, left and right, and those that touch corners diagonally), it dies.
\item But if it has more than three live neighbors, it also dies.
\item Having two or three neighbors, a cell lives on to the next generation.
\item If an empty square has exactly three neighbors, a new cell is born there.
\end{enumerate}

The rules are repeatedly applied, and one of two kinds of outcomes are possible: the entire population could die out, or the population settles into a periodic pattern that can go on infinitely.
%%%%%
%%%%%
\end{document}
