\documentclass[12pt]{article}
\usepackage{pmmeta}
\pmcanonicalname{Strategy}
\pmcreated{2013-03-22 12:52:02}
\pmmodified{2013-03-22 12:52:02}
\pmowner{Henry}{455}
\pmmodifier{Henry}{455}
\pmtitle{strategy}
\pmrecord{7}{33204}
\pmprivacy{1}
\pmauthor{Henry}{455}
\pmtype{Definition}
\pmcomment{trigger rebuild}
\pmclassification{msc}{91A99}
\pmrelated{Game}
\pmdefines{strategy}
\pmdefines{pure strategy}
\pmdefines{mixed strategy}
\pmdefines{strategy space}

\endmetadata

% this is the default PlanetMath preamble.  as your knowledge
% of TeX increases, you will probably want to edit this, but
% it should be fine as is for beginners.

% almost certainly you want these
\usepackage{amssymb}
\usepackage{amsmath}
\usepackage{amsfonts}

% used for TeXing text within eps files
%\usepackage{psfrag}
% need this for including graphics (\includegraphics)
%\usepackage{graphicx}
% for neatly defining theorems and propositions
%\usepackage{amsthm}
% making logically defined graphics
%%%\usepackage{xypic}

% there are many more packages, add them here as you need them

% define commands here
\begin{document}
A \emph{pure strategy} provides a \PMlinkescapetext{complete} definition for a way a player can play a game.  In particular, it defines, for every possible choice a player might have to make, which option the player picks.  A player's strategy space is the set of pure strategies available to that player.

A \emph{mixed strategy} is an assignment of a probability to each pure strategy.  It defines a probability over the strategies, and reflect that, rather than choosing a particular pure strategy, the player will randomly select a pure strategy based on the distribution given by their mixed strategy. Of course, every pure strategy is a mixed strategy (the function which takes that strategy to $1$ and every other one to $0$).

The following notation is often used:
\begin{itemize}

\item $S_i$ for the strategy space of the $i$-th player

\item $s_i$ for a particular element of $S_i$; that is, a particular pure strategy

\item $\sigma_i$ for a mixed strategy.  Note that $\sigma_i\in S_i\rightarrow [0,1]$ and $\sum_{s_i\in S_i} \sigma_i(s_i)=1$.

\item $\Sigma_i$ for the set of all possible mixed strategies for the $i$-th player

\item $S$ for $\prod_i S_i$, the set of all possible \PMlinkescapetext{combinations} of pure strategies (essentially the possible outcomes of the game)

\item $\Sigma$ for $\prod_i \Sigma_i$

\item $\sigma$ for a \emph{strategy profile}, a single element of $\Sigma$

\item $S_{-i}$ for $\prod_{j\neq i} S_j$ and $\Sigma_{-i}$ for $\prod_{j\neq i} \Sigma_j$, the sets of possible pure and mixed strategies for all players other than $i$.

\item $s_{-i}$ for an element of $S_{-i}$ and $\sigma_{-i}$ for an element of $\Sigma_{-i}$.

\end{itemize}
%%%%%
%%%%%
\end{document}
