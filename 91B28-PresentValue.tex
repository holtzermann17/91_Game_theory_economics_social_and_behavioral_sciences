\documentclass[12pt]{article}
\usepackage{pmmeta}
\pmcanonicalname{PresentValue}
\pmcreated{2013-03-22 16:40:59}
\pmmodified{2013-03-22 16:40:59}
\pmowner{CWoo}{3771}
\pmmodifier{CWoo}{3771}
\pmtitle{present value}
\pmrecord{9}{38893}
\pmprivacy{1}
\pmauthor{CWoo}{3771}
\pmtype{Definition}
\pmcomment{trigger rebuild}
\pmclassification{msc}{91B28}
\pmdefines{net present value}
\pmdefines{future value}

\endmetadata

\usepackage{amssymb,amscd}
\usepackage{amsmath}
\usepackage{amsfonts}
\usepackage{tabls}

% used for TeXing text within eps files
%\usepackage{psfrag}
% need this for including graphics (\includegraphics)
%\usepackage{graphicx}
% for neatly defining theorems and propositions
%\usepackage{amsthm}
% making logically defined graphics
%%\usepackage{xypic}
\usepackage{pst-plot}
\usepackage{psfrag}

% define commands here

\begin{document}
Suppose you are going to receive $\$10,000$, to be paid in two payments at the end of the next two years.  You have the following two options
\begin{center}
\begin{tabular}{|c||c|c|}
\hline  options & year 1 & year 2 \\
\hline\hline option 1 & $\$6,000$ & $\$4,000$ \\
\hline option 2 & $\$4,000$ & $\$6,000$  \\
\hline
\end{tabular}
\end{center}
Which option would you select in order to have the maximum gain?  Of course, if  there is no interest, both options are equal.  If any non-zero interest rates are involved, one option may be preferable than the other.

By calculating the \emph{present values} of these options, one may be able to compare the ``present'' values of these payments and figure out which is the preferable option.  So what is a \emph{present value}?

\textbf{Definition}.  Let $P$ be the amount of a payment at sometime $t>0$ in the future.  then the \emph{present value} $\operatorname{PV}(P)$ of $P$ is simply the value of this payment at time $t=0$.  Specifically, if the interest rate from $0$ to $t$ is $r$, then $$\operatorname{PV}(P)=\frac{P}{1+r}.$$
In other words, if we invest $\operatorname{PV}(P)$ today, earning an interest at a rate of $r$ between times $0$ and $t$, then at time $t$, we would have made $P$.

Now, suppose in the example above, both options have an \PMlinkname{effective annual interest rate}{InterestRate} of $5\%$ \PMlinkname{compounded annually}{CompoundInterest}, then the present value of option 1 is
$$\frac{\$6,000}{1.05}+\frac{\$4,000}{(1.05)^2}\approx \$9,342.40$$
whereas the second option has present value
$$\frac{\$4,000}{1.05}+\frac{\$6,000}{(1.05)^2}\approx \$9,251.70$$
Clearly, the first option is superior than the second one.

\textbf{Remarks}.
\begin{itemize}
\item
Of course, the result will be the same if one instead computes the \emph{future values} of these options, which are the values of the payments at a specific future time $t>0$: if payment is valued at $P$ at time $0$, its value at some future time $t>0$, or its \emph{future value} is 
$$\operatorname{FV}(P)=P(1+r),$$ if $r$ is the interest rate from $0$ to $t$.
\item
An accompanying concept is that of the \emph{net present value} $\operatorname{NPV}$.  It is the present value of all the future payments minus the initial investment: suppose an investment $I$ is made where an initial amount of $A$ is made at time $0$, and payments $P_1,\ldots,P_n$ are returns as a result of this investment.  Then 
$$\operatorname{NPV}(I)=\Big(\operatorname{PV}(P_1)+\operatorname{PV}(P_2)+\cdots  +\operatorname{PV}(P_n)\Big)-A.$$
\end{itemize}
If we treat the initial invsetment $A$ as a ``negative'' return, $A=-P_0=-\operatorname{PV}(P_0)$, then the net present value of the investment can be written
$$\operatorname{NPV}(I)=\operatorname{PV}(P_0)+\operatorname{PV}(P_1)+\cdots  +\operatorname{PV}(P_n)=\sum_{i=0}^n \operatorname{PV}(P_i).$$
One would usually want to invest in something with a positive net present value.  Net present values are commonly used when one is interested in comparing car loans or home mortgages.
%%%%%
%%%%%
\end{document}
