\documentclass[12pt]{article}
\usepackage{pmmeta}
\pmcanonicalname{RateOfReturn}
\pmcreated{2013-03-22 16:41:05}
\pmmodified{2013-03-22 16:41:05}
\pmowner{CWoo}{3771}
\pmmodifier{CWoo}{3771}
\pmtitle{rate of return}
\pmrecord{5}{38896}
\pmprivacy{1}
\pmauthor{CWoo}{3771}
\pmtype{Definition}
\pmcomment{trigger rebuild}
\pmclassification{msc}{91B28}

\endmetadata

\usepackage{amssymb,amscd}
\usepackage{amsmath}
\usepackage{amsfonts}

% used for TeXing text within eps files
%\usepackage{psfrag}
% need this for including graphics (\includegraphics)
%\usepackage{graphicx}
% for neatly defining theorems and propositions
%\usepackage{amsthm}
% making logically defined graphics
%%\usepackage{xypic}
\usepackage{pst-plot}
\usepackage{psfrag}

% define commands here

\begin{document}
Suppose you invest $P$ at time $0$ and receive payments $P_1,\ldots, P_n$ at times $t_1,\ldots,t_n$ corresponding to interest rates (evaluated from $0$) $r_1,\ldots, r_n$.  The net present value of this investment is 
$$NPV=-P+\frac{P_1}{1+r_1}+\frac{P_2}{1+r_2}+\cdots+\frac{P_n}{1+r_n}.$$

The \emph{rate of return} $r$ of this investment is a compound interest rate, compounded at every unit time period, such that the net present value of the investment is $0$.  In other words, if $r$, as a real number, exists, it satisfies the following equation:
$$P=\frac{P_1}{(1+r)^{t_1}}+\frac{P_2}{(1+r)^{t_2}}+\cdots+\frac{P_n}{(1+r)^{t_n}}.$$

\textbf{Remarks.}
\begin{itemize}
\item We typically assume that\, $t_1\le t_2\le \cdots \le t_n$,\, and, in most situations, that they are integers, so that the equation is a polynomial equation.
\item However, there is no guarantee that $r$ exists, and if it exists, that it is unique.
\item Nevertheless, one can usually, by trial-and-error, determine if such an $r$ exists.  If $r$ exists, and if $P_i$ are all non-negative, then by \PMlinkname{Descartes' rule of signs}{DescartesRuleOfSigns}, $r$ is always unique and $r>-1$.
\end{itemize}
%%%%%
%%%%%
\end{document}
