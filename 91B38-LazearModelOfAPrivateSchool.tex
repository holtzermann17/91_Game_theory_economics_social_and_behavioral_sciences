\documentclass[12pt]{article}
\usepackage{pmmeta}
\pmcanonicalname{LazearModelOfAPrivateSchool}
\pmcreated{2013-03-22 16:09:36}
\pmmodified{2013-03-22 16:09:36}
\pmowner{Mathprof}{13753}
\pmmodifier{Mathprof}{13753}
\pmtitle{Lazear model of a private school}
\pmrecord{8}{38241}
\pmprivacy{1}
\pmauthor{Mathprof}{13753}
\pmtype{Definition}
\pmcomment{trigger rebuild}
\pmclassification{msc}{91B38}

% this is the default PlanetMath preamble.  as your knowledge
% of TeX increases, you will probably want to edit this, but
% it should be fine as is for beginners.

% almost certainly you want these
\usepackage{amssymb}
\usepackage{amsmath}
\usepackage{amsfonts}

% used for TeXing text within eps files
%\usepackage{psfrag}
% need this for including graphics (\includegraphics)
%\usepackage{graphicx}
% for neatly defining theorems and propositions
%\usepackage{amsthm}
% making logically defined graphics
%%%\usepackage{xypic}

% there are many more packages, add them here as you need them

% define commands here

\begin{document}
\PMlinkescapeword{type}
\PMlinkescapeword{types}
\PMlinkescapeword{extension}
\PMlinkescapeword{unit}
\PMlinkescapeword{model}
\PMlinkescapeword{class}
\PMlinkescapeword{size}
\PMlinkescapeword{place}
\PMlinkescapeword{classes}
\PMlinkescapeword{fixed}


The economist Edward P. Lazear defined a theoretical \PMlinkescapetext{model} of a (private) profit
maximizing school and was able to derive some interesting conclusions from
the model. Here we describe the model.

The school has a fixed number, $Z$ of students. 
The cost of a teacher is $W>0$.
Each student is  assumed to be not disruptive with a probability $p$. So if
the class has $n$ students the probability of "peace in the classroom"
is $p^n$. Only when there is peace can any learning take place.
The school charges according the class size and all students in the class
of size $n$ pay the same amount. Assuming that the value of a unit of 
learning is $V>0$ and that there are $m$ classes, the average class
size is $n=Z/m$. The expected revenue then is 
$Vmn p^{Z/m} =VZp^{Z/m}$. And the cost for these $m$ classes is $Wm$.
Then the expected profit is $VZp^{Z/m}-Wm$. This is a function to be maximized.

Using this theoretical framework Lazear was able to show:
\begin{enumerate}
\item
As $p$ increases it is optimal to use fewer classes, so that class size increases.
\item
As $W$ increases it is optimal to use fewer classes, so that class size increases.
\end{enumerate}
Lazear briefly considered an extension of the model to the case where
there are two types of students, having different disruption probabilities. 
He concluded that if the school wants to maximize revenue it would segregate the 
students, that is, there would be no mixed classes. 

\begin{thebibliography}{10}
\bibitem[LAZ]{LAZ}
{\scshape Edward P. Lazear}, \emph{Educational Production}, Quarterly Journal of Economics, CXVI(2001), 777-801.
\end{thebibliography}
%%%%%
%%%%%
\end{document}
