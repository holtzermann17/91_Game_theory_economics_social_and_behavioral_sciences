\documentclass[12pt]{article}
\usepackage{pmmeta}
\pmcanonicalname{ConceptsOfVoting}
\pmcreated{2013-03-22 16:10:22}
\pmmodified{2013-03-22 16:10:22}
\pmowner{Mathprof}{13753}
\pmmodifier{Mathprof}{13753}
\pmtitle{concepts of voting}
\pmrecord{12}{38258}
\pmprivacy{1}
\pmauthor{Mathprof}{13753}
\pmtype{Topic}
\pmcomment{trigger rebuild}
\pmclassification{msc}{91F10}

% this is the default PlanetMath preamble.  as your knowledge
% of TeX increases, you will probably want to edit this, but
% it should be fine as is for beginners.

% almost certainly you want these
\usepackage{amssymb}
\usepackage{amsmath}
\usepackage{amsfonts}

% used for TeXing text within eps files
%\usepackage{psfrag}
% need this for including graphics (\includegraphics)
%\usepackage{graphicx}
% for neatly defining theorems and propositions
%\usepackage{amsthm}
% making logically defined graphics
%%%\usepackage{xypic}

% there are many more packages, add them here as you need them

% define commands here

\begin{document}
Various voting systems have been proposed and this entry lists the 
possible systems. 
\begin{enumerate}
\item
Single winner
\item
Range Voting
\item
Approval Voting
\item
Plurality Voting
\item
Strategic Voting
\item
Rank-order based
\item
Condorcet
\item
Instant Runoff
\item
Borda
\end{enumerate}

In addition there are various Impossibilty Theorems:
\begin{enumerate}
\item
\PMlinkescapetext{Arrow}
\item
Gibbard-Satterthwaite
\end{enumerate}

And there are various \PMlinkescapetext{terms} that are important for understanding Voting systems:
\begin{enumerate}
\item
No show paradox
\item
Alabama paradox
\item
Bayesian regret
\item
clone of a candidate
\end{enumerate}
%%%%%
%%%%%
\end{document}
