\documentclass[12pt]{article}
\usepackage{pmmeta}
\pmcanonicalname{EquivalentGrammars}
\pmcreated{2013-03-22 17:37:17}
\pmmodified{2013-03-22 17:37:17}
\pmowner{CWoo}{3771}
\pmmodifier{CWoo}{3771}
\pmtitle{equivalent grammars}
\pmrecord{6}{40041}
\pmprivacy{1}
\pmauthor{CWoo}{3771}
\pmtype{Definition}
\pmcomment{trigger rebuild}
\pmclassification{msc}{91F20}
\pmclassification{msc}{68Q42}
\pmclassification{msc}{68Q45}
\pmclassification{msc}{03D05}

\usepackage{amssymb,amscd}
\usepackage{amsmath}
\usepackage{amsfonts}
\usepackage{mathrsfs}

% used for TeXing text within eps files
%\usepackage{psfrag}
% need this for including graphics (\includegraphics)
%\usepackage{graphicx}
% for neatly defining theorems and propositions
\usepackage{amsthm}
% making logically defined graphics
%%\usepackage{xypic}
\usepackage{pst-plot}
\usepackage{psfrag}

% define commands here
\newtheorem{prop}{Proposition}
\newtheorem{thm}{Theorem}
\newtheorem{ex}{Example}
\newcommand{\real}{\mathbb{R}}
\newcommand{\pdiff}[2]{\frac{\partial #1}{\partial #2}}
\newcommand{\mpdiff}[3]{\frac{\partial^#1 #2}{\partial #3^#1}}
\begin{document}
\PMlinkescapeword{equivalent}
\PMlinkescapeword{generate}


Two formal grammars $G_1=(\Sigma_1,N_1,P_1,\sigma_1)$ and $G_2=(\Sigma_2,N_2,P_2,\sigma_2)$ are said to be \emph{equivalent} if they generate the same formal languages: $$L(G_1)=L(G_2).$$
When two grammars $G_1$ and $G_2$ are equivalent, we write $G_1\cong G_2$.

For example, the language $\lbrace a^nb^{2n}\mid n\mbox{ is a non-negative integer}\rbrace$ over $T=\lbrace a,b,c\rbrace$ can be generated by a grammar $G_1$ with three non-terminals $x,y,\sigma$, and the following productions:
$$P_1=\lbrace \sigma \to \lambda, \sigma \to x\sigma y, x\to a, y\to b^2 \rbrace.$$
Alternatively, it can be generated by a simpler grammar $G_2$ with just the starting symbol:
$$P_2=\lbrace \sigma \to \lambda, \sigma \to a\sigma b^2\rbrace.$$
This shows that $G_1 \cong G_2$.

An alternative way of writing a grammar $G$ is $(T,N,P,\sigma)$, where $T=\Sigma-N$ is the set of terminals of $G$.  Suppose $G_1=(T_1,N_1,P_1,\sigma_1)$ and $G_2=(T_2,N_2,P_2,\sigma_2)$ are two grammars.  Below are some basic properties of equivalence of grammars: 
\begin{enumerate}
\item
Suppose $G_1\cong G_2$.  If $T_1\cap T_2=\varnothing$, then $L(G_1)=\varnothing$.
\item
Let $G=(T,N,P,\sigma)$ be a grammar.  Then the grammar $G'=(T,N',P,\sigma)$ where $N\subseteq N'$ is equivalent to $G$.
\item
If $(T_1,N_1,P_1,\sigma_1)\cong (T_1,N_1,P_2,\sigma_2)$, then $(T,N,P_1,\sigma_1)\cong (T,N,P_2,\sigma_2)$, where $T=T_1\cup T_2$ and $N=N_1\cup N_2$.
\item
If we fix an alphabet $\Sigma$, then $\cong$ is an equivalence relation on grammars over $\Sigma$.
\end{enumerate}

So far, the properties have all been focused on the underlying alphabets.  More interesting properties on equivalent grammars center on productions.  Suppose $G=(\Sigma,N,P,\sigma)$ is a grammar.
\begin{enumerate}
\item
A production is said to be trivial if it has the form $v\to v$.  Then the grammar $G'$ obtained by adding trivial productions to $P$ is equivalent to $G$.
\item
If $u\in L(G)$, then adding the production $\sigma\to u$ to $P$ produces a grammar equivalent to $G$.
\item
Call a production a \emph{filler} if it has the form $\lambda \to v$.  Replacing a filler $\lambda \to v$ by productions $a\to va$ and $a\to av$ using all symbols $a\in \Sigma$ gives us a grammar that is equivalent to $G$.
\item
$G$ is equivalent to a grammar $G'$ without any fillers.  This can be done by replacing each filler using the productions mentioned in the previous section.  Finally, if $\lambda\in L(G)$, we add the production $\sigma \to \lambda$ to $P$ if it were not already in $P$.
\item
Let $S(P)$ be the set of all symbols that occur in the productions in $P$ (in either antecedents or conseqents).  If $a\in S(P)$, then $G'$ obtained by replacing every production $u\to v \in P$ with production $u[a/b]\to v[a/b]$ and $b\to a$, with a symbol $X\notin \Sigma$, is equivalent to $G$.  Here, $u[a/b]$ means that we replace each occurrence of $a$ in $u$ with $X$.
\item
$G$ is equivalent to a grammar $G'$, all of whose productions have antecedents in $N^+$ (non-empty words over $N$).
\end{enumerate}

\textbf{Remark}.  In fact, if we let 
\begin{enumerate}
\item $X\to XY$,
\item $XY\to ZT$,
\item $X\to \lambda$,
\item $X\to a$
\end{enumerate}
be four types of productions, then it can be shown that 
\begin{itemize}
\item any formal grammar is equivalent to a grammar all of whose productions are in any of the four forms above
\item any context-sensitive grammar is equivalent to a grammar all of whose productions are in any of forms 1, 2, or 4, and
\item and any context-free grammar is equivalent to a grammar all of whose productions are in any of forms 1, 3, or 4.
\end{itemize}

\begin{thebibliography}{9}
\bibitem{as} A. Salomaa {\em Computation and Automata, Encyclopedia of Mathematics and Its Applications, Vol. 25}. Cambridge (1985).
\end{thebibliography}
%%%%%
%%%%%
\end{document}
